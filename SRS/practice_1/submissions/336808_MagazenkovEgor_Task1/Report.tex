\documentclass[a4paper, 14pt]{extarticle}

% Текст
\usepackage[utf8]{inputenc} % UTF-8 кодировка
\usepackage[russian]{babel} % Русский язык
\usepackage{indentfirst} % красная строка в первом параграфе в главе
% Отображение страниц
\usepackage{geometry} % размеры листа и отступов
\geometry{
	left=30mm,
	top=20mm,
	right=15mm,
	bottom=25mm,
	marginparsep=0mm,
	marginparwidth=0mm,
	headheight=10mm,
	headsep=7mm,
	foot=0mm}
\usepackage{afterpage,fancyhdr} % настройка колонтитулов

\setlength{\baselineskip}{1.5em}
\usepackage{titlesec}
\renewcommand{\thesection}{}
\renewcommand{\thesubsection}{\arabic{subsection}}
\titleformat{\section}[block]{\centering\bfseries\Large}{}{5pt}{}
\titleformat{\subsection}[block]{\bfseries\large}{}{5pt}{}

\pagestyle{fancy}
\fancypagestyle{style}{ % создание нового стиля style
	\fancyhf{} % очистка колонтитулов
    \fancyhead[LO, RE]{\nouppercase{ИМРС}} % название документа наверху
    \fancyhead[RO, LE]{\nouppercase{\leftmark}} % название section наверху
	\fancyfoot[C]{\thepage} % номер страницы справа внизу на нечетных и слева внизу на четных
	\renewcommand{\headrulewidth}{0.25pt} % толщина линии сверху
	\renewcommand{\footrulewidth}{0pt} % толцина линии снизу
}
\fancypagestyle{plain}{ % создание нового стиля plain -- полностью пустого
	\fancyhf{}
	\renewcommand{\headrulewidth}{0pt}
}
\fancypagestyle{title}{ % создание нового стиля title -- для титульной страницы
	\fancyhf{}
	\fancyhead[C]{{\footnotesize
			Министерство образования и науки Российской Федерации\\
			Федеральное государственное автономное образовательное учреждение высшего образования
	}}
	\fancyfoot[C]{{\large 
			Санкт-Петербург, 2024-2025
	}}
	\renewcommand{\headrulewidth}{0pt}
}

% Математика
\usepackage{amsmath, amsfonts, amssymb, amsthm} % Набор пакетов для математических текстов
\usepackage{cancel} % зачеркивание для сокращений
% Рисунки и фигуры
\usepackage{graphicx} % вставка рисунков
\usepackage{epstopdf}
\usepackage{wrapfig, subcaption} % вставка фигур, обтекая текст
\usepackage{caption} % для настройки подписей
\captionsetup{figurewithin=none,labelsep=period, font={small,it}} % настройка подписей к рисункам
% Рисование
\usepackage{tikz} % рисование
\usepackage{circuitikz}
\usepackage{pgfplots} % графики
\usepgfplotslibrary{fillbetween}
% Таблицы
\usepackage{multirow} % объединение строк
\usepackage{multicol} % объединение столбцов
% Остальное
\usepackage[unicode, pdftex]{hyperref} % гиперссылки
\usepackage{enumitem} % нормальное оформление списков
\usepackage{float}

\setlist{itemsep=0.15cm,topsep=0.15cm,parsep=1pt} % настройки списков
% Теоремы, леммы, определения...
\theoremstyle{definition}
\newtheorem{Def}{Определение}
\newtheorem*{Axiom}{Аксиома}
\theoremstyle{plain}
\newtheorem{Th}{Теорема}
\newtheorem{Task}{Задание}
\newtheorem{Lem}{Лемма}
\newtheorem{Cor}{Следствие}
\newtheorem{Ex}{Пример}
\theoremstyle{remark}
\newtheorem*{Note}{Замечание}
\newtheorem*{Solution}{Решение}
\newtheorem*{Proof}{Доказательство}
% Свои команды
\newcommand{\comb}[1]{\left[\hspace{-4pt}\begin{array}{l}#1\end{array}\right.\hspace{-5pt} } % совокупность уравнений
\newcommand{\rank}{\mathrm{rank}\;}
% Титульный лист

\usepackage{listings}
\newcommand*{\titlePage}{
	\thispagestyle{title}
	\begingroup
	\begin{center}
		%		{\footnotesize
			%			Министерство образования и науки Российской Федерации\\
			%			Федеральное государственное автономное образовательное учреждение высшего образования
			%		}
		%		
		\vspace*{3ex}
		{\small
			САНКТ-ПЕТЕРБУРГСКИЙ НАЦИОНАЛЬНЫЙ ИССЛЕДОВАТЕЛЬСКИЙ УНИВЕРСИТЕТ ИНФОРМАЦИОННЫХ ТЕХНОЛОГИЙ, МЕХАНИКИ И ОПТИКИ	
		}
		
		\vspace*{2ex}
		
		{\normalsize
			Факультет систем управления и робототехники
		}
		
		\vspace*{15ex}
		
		{\Large \bfseries 
			Отчёт по лабораторной работе №4 (8)\\
			{\large по теме <<Дифференциальные уравнения>>\\
				по дисциплине "Имитационное моделирование робототехнических систем"\\
				}
			
		}
		
	\end{center}
	\vspace*{10ex}
	\begin{flushright}
		{\large 
			\underline{Выполнил}: студент гр. \textbf{}\\
			\begin{flushright}
				\textbf{Магазенков Е. Н.}\\
			\end{flushright}
		}
		\vspace*{5ex}
		{\large 
			\underline{Преподаватели}:\\ 
			\begin{flushright}
            \textit{Борисов И. И., Ракшин Е. А.}
			\end{flushright}
		}
	\end{flushright}	
	\newpage
	\setcounter{page}{1}
	\endgroup}
%\usepackage{newtxmath,newtxtext}
%\lstset{literate={а}{\cyra}1{б}{\cyrb}1{в}{\cyrv}1{г}{\cyrg}1{д}{\cyrd}1{е}{\cyre}1{ж}{\cyrzh}1{з}{\cyrz}1{и}{\cyri}1{к}{\cyrk}1{л}{\cyrl}1{м}{\cyrm}1{н}{\cyrn}1{о}{\cyro}1{п}{\cyrp}1{р}{\cyrr}1{с}{\cyrs}1{т}{\cyrt}1{у}{\cyru}1{ф}{\cyrf}1{х}{h}1{ц}{w}1{ч}{\cyrch}1{ш}{\cyrsh}1{щ}{\cyrshch}1{ь}{m}1{ъ}{m}1{ы}{y}1{э}{e}1{ю}{\cyryu}1{я}{\cyrya}}

\lstset{basicstyle=\small}
\newcommand{\tasknum}[3]{Task}%\textunderscore{#1}\textunderscore{#2}y\textunderscore{#2}\textunderscore{#3}}
\usepackage{pdfpages}

\newcommand{\mat}[1]{\begin{pmatrix}#1\end{pmatrix}} 
\newcommand{\bmat}[1]{\begin{bmatrix}#1\end{bmatrix}} 

\newcommand{\code}[2]
{
\begin{minipage}{0.45\textwidth}
    \textbf{Code:}
    #1
\end{minipage}
}

\begin{document}
\renewcommand{\contentsname}{\hfillОГЛАВЛЕНИЕ\hfill} 
\titlePage
\thispagestyle{plain}
\tableofcontents
\pagestyle{style}

\newpage
\setcounter{page}{1}

% \includepdf[pages={33}, scale=1,addtotoc={1, section, 1, {Текст задания}, text}]{./Tasks.pdf}

\begin{table}
    \centering
    \caption{Вариант задания}
    \begin{tabular}
        {|c|c|c|c|}
        \hline 
        $a$ & $b$ & $c$ & $d$ \\
        \hline
        $-7$&	$-8.6$&	$1.66$& $8.12$\\
        \hline
    \end{tabular}
\end{table}
\section{Аналитическое решение уравнения}
Решим уравнение
\[
    a\ddot{x} + b\dot{x} + c x = d.
\] 
Для этого решим соответствующее однородное линейное уравнение:
\[
    a\ddot{x} + b\dot{x} + cx = 0.
\] 
Составим характеристическое уравнение
\[
    a\lambda^2 + b\lambda + c = 0 \Leftrightarrow \lambda_{12} = \frac{-b \pm \sqrt{b^2-4ac}}{2a}. 
\] 
Тогда решением однородного уравнения будет 
\[
    x = C_1 e^{\lambda_1 t} + C_2 e^{\lambda_2 t}.
\] 

Правая часть исходного неоднородного уравнения имеет константный вид, поэтому будем искать частное решение в виде $x = D$. Очевидно, получается, что $D = \frac{d}{c}$.

Таким образом, общее решение исходного уравнения имеет вид
\[
    x = C_1 e^{\lambda_1 t} + C_2 e^{\lambda_2 t} + \frac{d}{c}.
\] 

При подстановке значений из варианта, получаем уравнение
\[
    -7\ddot{x} - 8.6 \dot{x} + 1.66 x = 8.12
\] 
и $\lambda_1=0.1696$, $\lambda_2 = -1.3982$, а значит решение 
\[
    x = C_1 e^{0.1696t} + C_2 e^{-1.3982t} + 4.8916.
\] 

\section{Численное решение уравнения}
Численно будем решать в виде системы уравнений первого порядка. Для этого приведем исходное уравнение к виду
\[
    \underbrace{\mat{\dot x\\\ddot{x}}}_\mathbf{x} = \mat{0 & 1 \\ -\frac{c}{a} & -\frac{b}{a}} \mat{x \\ \dot{x}} + \mat{0\\ \frac{d}{a}}.
\] 

Рассмотрим методы:
\begin{itemize}
    \item прямого Эйлера
        \[
            \mathbf{x}_{k+1} = \mathbf{x}_k + h\cdot f(t_k, \mathbf{x}_k),
        \] 
    \item обратного Эйлера
        \[
            \mathbf{x}_{k+1} = \mathbf{x}_k + h\cdot f(t_{k+1}, \mathbf{x}_{k+1}),
        \] 
    \item Рунге-Кутты 4-ого порядка
        \[
            \mathbf{x}_{k+1} = \mathbf{x}_k + \frac{h}{6}\left(  k_1 + 2k_2 + 2k_3+k_4) \right),
        \] где $k_1= f(t_k, \mathbf{x}_k)$, $k_2= f(t_k+\frac{h}{2}, \mathbf{x}_k + \frac{h}{2} k_1)$, $k_3= f(t_k + \frac{h}{2}, \mathbf{x}_k + \frac{h}{2} k_2)$, $k_4= f(t_k + h, \mathbf{x}_k + h k_3)$.
\end{itemize}


\section{Результаты}
Промоделируем с произвольным зафиксированным начальным условием $x(0)=452, \dot{x}(0)=449$ и разными шагами $h\in \{0.1, 0.01, 0.001\}$. Результаты с графиками решений уравнения и графиками ошибок численного решения от аналитического представлены на рисунках \ref{1}-\ref{3}. 
\begin{figure}
    [H]
    \centering
    \includegraphics[width=1\textwidth]{0.1}
    \caption{Решение уравнения с шагом $h=0.1$ }
    \label{1}
\end{figure}
\begin{figure}
    [H]
    \centering
    \includegraphics[width=1\textwidth]{0.01}
    \caption{Решение уравнения с шагом $h=0.01$ }
    \label{2}
\end{figure}
\begin{figure}
    [H]
    \centering
    \includegraphics[width=1\textwidth]{0.001}
    \caption{Решение уравнения с шагом $h=0.001$ }
    \label{3}
\end{figure}

\section{Выводы}
Из результатов видно подтверждение теоретических знаний об ошибках численных методов:
\begin{itemize}
    \item С уменьшением величины шага $h$ уменьшается ошибка решения.
    \item При этом у метода Рунге-Кутты ошибка сильно меньше, так как он четвертого порядка и имеет накопительную ошибку четвертого порядка от величины шага, в то время как методы прямого и обратного Эйлера имеют ошибку первого порядка (то есть сопоставимую с величиной шага).
    \item Также стоит отметить, что метод обратного Эйлера в данном уравнении не дает преимущества, а только увеличивает вычислительную сложность в силу своей неявности. В жестких системах он будет необходим, но тут не дает преимущества.
\end{itemize}

\end{document}

