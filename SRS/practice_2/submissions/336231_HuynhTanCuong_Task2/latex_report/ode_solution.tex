\chapter{Solution}

To write the equations of motion, we begin by deriving the Lagrangian for the system, $L$, as the difference between the kinetic and potential energy of the system
\[
L = K - P
\]
where $K$ is the kinetic energy and $P$ is the potential energy of the system.
\[
K = \frac{1}{2}m\omega^2 = \frac{1}{2} m l^2 \dot{\theta}^2
\]
\[
P = mgl(1-\cos\theta) + \frac{1}{2} k \theta^2
\]

The Lagrange's equations of motion:
\[
\frac{d}{dt} \left( \frac{\partial L}{\partial \dot{\theta}} \right) - \frac{\partial L}{\partial \theta} = Q,
\]
where $Q$ is the damping force in our case $Q = -b \dot{\theta}$

Let's compute these parts of the equation: \\
The partial derivative of $L$ in respect to $\theta$
\[
\frac{\partial L}{\partial \theta} = -mgl\sin\theta - k\theta
\]
The partial derivative of $L$ in respect to $\dot{\theta}$
\[
\frac{\partial L}{\partial \dot{\theta}} = ml^2\dot{\theta}
\]
the full time derivative
\[
\frac{d}{dt} \left( \frac{\partial L}{\partial \dot{\theta}} \right)
=
ml^2 \ddot{\theta} 
\]

Finally, the equations of motion for a pendulum-spring-damper system
\[
ml^2\ddot{\theta} + mgl\sin\theta + k\theta = -b\dot{\theta}
\]
or it can rewritten in the next form
\[
\ddot{\theta} = -\frac{g}{l}\sin\theta - \frac{k}{ml^2}\theta - \frac{b}{ml^2}\dot{\theta}
\]

This differential equation is nonlinear due to the part $\sin\theta$. The equation can be linearized ($\sin\theta \approx \theta$) only if the initial angle of the pendulum is small $\theta_0 < 5 \deg$. However, in my case, the initial angle $\theta_0 \approx -90 \deg$, so we can not do it. The analytical solution is unavailable, but we can solve it using numerical methods. 


\section{Simulation result}

The simulation results show the behavior of the system in 10 second using 3 integrator methods: Forward Euler, Backward Euler and Runge-Kutta (RK4). As we can observe: 
\begin{itemize}
\item When the time step is small (see \ref{fig:simdt0}), 3 methods is stable 
\item When the time step is medium (see \ref{fig:simdt1}), the Forward Euler tend to increase the amplitude of the motion. The Backward Euler decrease the amplitude of the motion. But the RK4 is still stable.
\item When the time step is big (see \ref{fig:simdt2}), the Forward Euler become unstable.
\end{itemize}


\subsection{Case study with small time step (dt=0.001)}
\begin{figure}[H]
\centering
\includegraphics[width=1.0\linewidth]{images/simdt0.001}
\caption{Simulation result with small time step}
\label{fig:simdt0}
\end{figure}

\subsection{Case study with medium time step (dt=0.01)}
\begin{figure}[H]
\centering
\includegraphics[width=1.0\linewidth]{images/simdt0.01}
\caption{Simulation result with medium time step}
\label{fig:simdt1}
\end{figure}

\subsection{Case study with large time step (dt=0.1)}
\begin{figure}[H]
\centering
\includegraphics[width=1.0\linewidth]{images/simdt0.1}
\caption{Simulation result with large time step}
\label{fig:simdt2}
\end{figure}

\chapter{Conclusion}

In this lab, we have derive the equation of motion for a pendulum-spring-damping system using Lagrangian method. Then we conduct the simulation using 3 different integrator methods with different time step. The result shows that method RK4 is the most stable and accurate.   




